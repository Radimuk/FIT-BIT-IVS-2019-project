\documentclass[12pt]{article}
\usepackage[a4paper, text={17cm,24cm}, top=3cm, left=2cm]{geometry}
\usepackage[utf8]{inputenc}
\usepackage{times}
\usepackage[czech]{babel}
\usepackage[unicode]{hyperref}
\usepackage{graphicx}
\begin{document}

\begin{titlepage}
	\begin{center}

		\includegraphics[height = 96pt]{img/FIT_barevne_CMYK_CZ.pdf} \\

		\begin{LARGE}
			\textbf{Vysoké učení technické v Brně} \\
		\end{LARGE}

		\begin{large}
			Fakulta informačních technologií \\
			Elektronika pro informační technologie \\
			2018~/~2019 
		\end{large}
		\\[78mm]

		\begin{huge}
				\textbf{Uživatelská dokumentace} \\
			\begin{large}		
		Radim Lipka xlipka02@stud.fit.vutbr.cz
		
        Roman Ondráček xondra58@stud.fit.vutbr.cz
        
        Pavel Raur xraurp00@stud.fit.vutbr.cz
        
        David Reinhart xreinh00@stud.fit.vutbr.cz
        \end{large}
		\end{huge}
	\end{center}

	\vfill


\end{titlepage}

\tableofcontents
\newpage

\section{Instalace kalkulačky}
Existují dva způsoby, jak kalkulačku nainstalovat, jedna možnost je nainstalovat ji manuálně po stažení z repozitáře, druhá možnost je stáhnout si ji a nainstalovat z balíčků pro Debian.
\subsection{Instalace z repozitáře}
Pokud si chcete nainstalovat kalkulačku z repozitáře, stačí navšívit stránku \\ 
\href{https://github.com/Roman3349/FIT-BIT-IVS-2019-project}{https://github.com/Roman3349/FIT-BIT-IVS-2019-project} a kliknout na \textbf{Download} nebo-li stáhnout. Po určení místa uložení pak stačí ve staženém adresáři napsat \textbf{sudo make install} a kalkulačka se nainstaluje. Pro odinstalovaní potom slouží příkaz \textbf{sudo make uninstall}.
\subsection{Instalace z balíčku}
Pokud se rozhodnete pro nainstalování si kalkulačky z balíčků pro debian, je nutné provést následující sekvenci příkazů. 
\\
if [ "\$(whoami)" != "root" ]; then \\
SUDO=sudo \\
fi 
\begin{itemize}
    \item \${SUDO} apt-get -y install apt-transport-https lsb-release ca-certificates
    \item \${SUDO} wget -O /etc/apt/trusted.gpg.d/romanondracek.gpg\\ https://deb.romanondracek.cz/ubuntu/apt.gpg 
    \item \${SUDO} sh -c 'echo "deb https://deb.romanondracek.cz/ubuntu/\\  \$(lsb\_release -sc) main" > /etc/apt/sources.list.d/romanondracek.list'
    \item \${SUDO} apt-get update
    \item a nakonec sudo apt-get install fit-calc
\end{itemize}

\section{Orientace v prostředí kalkulačky}
\subsection{Horní lišta}
V horní liště kalkulačky můžete vidět tlačítko \textbf{Nápověda}.

Při kliknutí na toto tlačítko se zobrazí možnost \textbf{O aplikaci}, kde můžete zjistit informace o aktuální verzi kalkulačky, odkaz na GitHub repozitář, seznam tvůrců a licenci, pod kterou je kalkulačka šířená a možnost \textbf{Nápověda}, kde si můžete přečíst nápovědu k používání kalkulačky.
\subsection{Prostředí kalkulačky}
Samotné prostřední kalkulačky je rozděleno do tří sekcí.

Úplně nahoře můžete vidět okno, ve kterém se zobrazují vámi zadávané hodnoty, například pro kontrolu, jestli jste při zadávání neudělali nějakej chybu, která by ovlivnila výsledek výpočtu.

Přibližně uprostřed se nachází řádek sloužící pro váš vstup do kalkulačky.

A nakonec ve spodní části je k vidění klávesnice s čísly a různými funkcemi pro výpočet.

\section{Ovládání kalkulačky}
Vstup do kalkulačky lze zadat pomocí klikání na příslušná tlačítka. Čísla a některé funkce (pokud máte příslušné symboly na klávesnici) lze zadat i z klávesnice.
\begin{itemize}
\item Pro zadání čísel 0-9 slouží klávesy 0-9.
\item Pro zadání konstanty PI a Eulerovova čísla použijte klavesy $\pi$ a $e$.
\item Pro použití základních matematických operací plus, minus, krát a děleno, slouží tlačítka +, -, $\times$ a $\div$.
\item Pro oddělelování celých a desetinných částí čísel slouží desetinná čárka $,$.
\item Pro počítání s mocninami a odmocninou slouží tlačítka $x^n$ pro obecnou mocninu, $x^{-1}$ pro získání inverzního čísla a $\sqrt{}$ pro získání odmocniny z určitého čísla. Pro n--tou odmocninu je syntax $n\sqrt{}$. Při zadání pouze $\sqrt{}$ získáte druhou mocninu čísla.
\item Pro počítání s procenty slouží symbol \%, přičemž syntax pro počítání s procenty je x +  (-, $\times$ nebo $\div$) y\%.
\item Pro zbytek po dělení modulo, logaritmus se základem deset a přirozený logaritmus slouží tlačítka $mod, log$ a $ln$.
\item Pro počítání s faktoriálem a absolutní hodnotou slouží tlačítka $x!$ a $|x|$.
\item Pro použití goniometrických funkcí sinus, cosinus a tangens slouží tlačítka $sin, cos$ a $tan$ v pravém sloupci.
\item Tlačítko C slouží pro vymazání celého vstupu. Tlačítko $\longleftarrow$ je určeno pro vymazání jednoho znaku ze vstupu.
\item Tlačítko Ans slouží pro zapamatování výsledku posledního výpočtu.
\end{itemize}



\end{document}