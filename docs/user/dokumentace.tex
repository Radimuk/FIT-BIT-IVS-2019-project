\documentclass[11pt,a4paper,titlepage]{article}
\usepackage[a4paper,top=3cm,left=2cm,right=2cm,bottom=2cm]{geometry}
\usepackage[utf8]{inputenc}
\usepackage[czech]{babel}
\usepackage[unicode]{hyperref}
\usepackage{graphicx}
\title{Uživatelská dokumentace}
\begin{document}

	\begin{titlepage}
		\begin{center}
			\includegraphics[height = 96pt]{img/FIT_barevne_CMYK_CZ.pdf} \\

			\begin{LARGE}
				\textbf{Vysoké učení technické v~Brně} \\
			\end{LARGE}

			\begin{Large}
				\textbf{Fakulta informačních technologií} \\
			\end{Large}

			\begin{large}
				Praktické aspekty vývoje software \\
				2018~/~2019
			\end{large}

			\vspace{\stretch{0.382}}

			\begin{huge}
				\textbf{Uživatelská dokumentace} \\
			\end{huge}

			\vspace{\stretch{0.618}}

			\begin{large}
				Radim Lipka xlipka02@stud.fit.vutbr.cz \\
				Roman Ondráček xondra58@stud.fit.vutbr.cz \\
				Pavel Raur xraurp00@stud.fit.vutbr.cz \\
				David Reinhart xreinh00@stud.fit.vutbr.cz \\
			\end{large}
		\end{center}
	\end{titlepage}


	\tableofcontents
	\newpage

	\section{Instalace}

	Existují tři způsoby, jak kalkulačku nainstalovat:

	\begin{itemize}
		\item kompilace a instalace ze zdrojových souborů - viz \ref{instalace:manualni},
		\item sestavení a instalace balíčku pro linuxové distribuce založené na Debianu - viz \ref{instalace:balicek},
		\item instalace z~repozitáře - viz \ref{instalace:repozitar}.
	\end{itemize}

	\subsection{Kompilace a instalace ze zdrojových souborů}\label{instalace:manualni}

	Nejdříve je nutné nainstalovat závislosti potřebné ke kompilaci ze zdrojových souborů. \\
	Na linuxových distribucích založených na Debianu se jedná o~tyto balíčky: \texttt{build-essential, cmake, git, gnome-icon-theme, libgtest-dev, libgtkmm-3.0-dev} a \texttt{uuid-dev}. \\

	Poté je nutné naklonovat repozitář pomocí následujícího příkazu \\ \texttt{git clone https://github.com/Roman3349/FIT-BIT-IVS-2019-project --recurse-submodules} \\ a přejít do složky \texttt{FIT-BIT-IVS-2019-project}. \\
	Kde poté stačí zadat příkaz \texttt{sudo make install} pro instalaci a pro odinstalaci stačí zadat příkaz \texttt{sudo make uninstall}.

	\subsection{Sestavení a instalace balíčku}\label{instalace:balicek}

	Najdříve je nutné nainstalovat závisloti potřebné ke kompilaci a také k~sestavení balíčku. \\
	Na linuxových distribucích založených na Debianu se jedná o~tyto balíčky: \texttt{build-essential, cmake, devscripts, git, git-buildpackage, gnome-icon-theme, libgtest-dev, libgtkmm-3.0-dev} a \texttt{uuid-dev}. \\

	Poté je nutné naklonovat repozitář pomocí následujícího příkazu \\ \texttt{git clone https://github.com/Roman3349/FIT-BIT-IVS-2019-project --recurse-submodules} \\ a přejít do složky \texttt{FIT-BIT-IVS-2019-project}. \\
	Kde poté stačí zadat příkaz \texttt{make deb-package} pro vytvoření balíčku. \\
	Poté stačí vytvořený balíček nainstalovat pomocí příkazu \texttt{sudo apt-get install ./../fit-calc\_*.deb}. \\
	Pro případnou odinstalaci stačí zadat příkaz \texttt{sudo apt-get purge fit-calc}.

	\subsection{Instalace z~repozitáře}\label{instalace:repozitar}

	Tento způsob instalace je aktuálně podporovaný pouze na linuxových distribucích Debian testing a Ubuntu 18.04. \\
	Nejdříve je nutné nainstalovat balíčky, které jsou nutné pro podporu HTTPS repozitářů pomocí příkazu \texttt{sudo apt-get -y install apt-transport-https lsb-release ca-certificates}. \\

	Poté je nutné stáhnout PGP klíč, pomocí kterého je repozitář podepsaný, pomocí příkazu: \\ {\scriptsize\texttt{sudo wget -O /etc/apt/trusted.gpg.d/romanondracek.gpg https://deb.romanondracek.cz/debian/apt.gpg}}. \\
	Dále přidáme repozitář do nastavení balíčkovacícho systému na Debianu pomocí příkazu: \\ {\scriptsize\verb~echo "deb https://deb.romanondracek.cz/debian/ testing main" | sudo tee /etc/apt/sources.list.d/romanondracek.list'~} případně na Ubuntu pomocí příkazu: \\ {\scriptsize\verb~echo "deb https://deb.romanondracek.cz/ubuntu/ bionic main" | sudo tee /etc/apt/sources.list.d/romanondracek.list'~}. \\
	Pak aktualizujeme seznam balíčků v~repozitářích pomocí příkazu \texttt{sudo apt-get update} a nakonec nainstalujeme kalkulačku pomocí příkazu \texttt{sudo apt-get install fit-calc}.  \\
	Pro případnou odinstalaci stačí zadat příkaz \texttt{sudo apt-get purge fit-calc}. \\

	\newpage

	\section{Orientace v~prostředí kalkulačky}

	\subsection{Horní lišta}

	V~horní liště kalkulačky můžete vidět tlačítko \textbf{Nápověda}.

	Při kliknutí na toto tlačítko se zobrazí možnost \textbf{O~aplikaci}, kde můžete zjistit informace o~aktuální verzi kalkulačky, odkaz na GitHub repozitář, seznam tvůrců a licenci, pod kterou je kalkulačka šířená a možnost \textbf{Nápověda}, kde si můžete přečíst nápovědu k~používání kalkulačky.

	\subsection{Prostředí kalkulačky}

	Samotné prostřední kalkulačky je rozděleno do tří sekcí.

	Úplně nahoře můžete vidět okno, ve kterém se zobrazují vámi zadávané hodnoty, například pro kontrolu, jestli jste při zadávání neudělali nějakej chybu, která by ovlivnila výsledek výpočtu.

	Přibližně uprostřed se nachází řádek sloužící pro váš vstup do kalkulačky.

	A~nakonec ve spodní části je k~vidění klávesnice s~čísly a různými funkcemi pro výpočet.

	\section{Ovládání kalkulačky}

	Vstup do kalkulačky lze zadat pomocí klikání na příslušná tlačítka. Čísla a některé funkce (pokud máte příslušné symboly na klávesnici) lze zadat i z~klávesnice.

	\begin{itemize}
		\item Pro zadání čísel 0-9 slouží klávesy 0-9.
		\item Pro zadání konstanty PI a Eulerovova čísla použijte klavesy $\pi$ a $e$.
		\item Pro použití základních matematických operací sčítání, odčítání, násobení a dělení, slouží tlačítka +, -, $\times$ a $\div$.
		\item Pro oddělelování celých a desetinných částí čísel slouží desetinná čárka $,$.
		\item Pro počítání s~mocninami a odmocninou slouží tlačítka $x^n$ pro obecnou mocninu, $x^{-1}$ pro získání inverzního čísla a $\sqrt{}$ pro získání odmocniny z~určitého čísla. Pro n--tou odmocninu je syntax $n\sqrt{}$. Při zadání pouze $\sqrt{}$ získáte druhou mocninu čísla.
		\item Pro počítání s~procenty slouží symbol \%, přičemž syntax pro počítání s~procenty je x +  (-, $\times$ nebo $\div$) y\%.
		\item Pro zbytek po dělení modulo, logaritmus se základem deset a přirozený logaritmus slouží tlačítka $mod, log$ a $ln$.
		\item Pro počítání s~faktoriálem a absolutní hodnotou slouží tlačítka $x!$ a $|x|$.
		\item Pro použití goniometrických funkcí sinus, cosinus a tangens slouží tlačítka $sin, cos$ a $tan$ v~pravém sloupci.
		\item Tlačítko C slouží pro vymazání celého vstupu. Tlačítko $\leftarrow$ je určeno pro vymazání jednoho znaku ze vstupu.
	\end{itemize}

\end{document}
